% This file contains macros that can be called up from connected TeX files
% It helps to summarise repeated code, e.g. figure insertion (see below).


\usepackage{listings}
\usepackage{booktabs}  % Nice tables
\usepackage{dsfont} % To use the R of the real numbers (and the N, Z, etc.)
\usepackage{longtable}

% Inserts a TODO field in your text. Useful when you need to remember
% future tasks.
% Parameters 1: the text associated to your TODO.
% Usage: \TODO{your_todo_text_here}
\newcommand{\TODO}[1] {\textcolor{magenta}{\textbf{TODO: #1}}}

% Track Changes Mode: strikes the first argument and replaces by the second one.
%\usepackage[normalem]{ulem}
%\usepackage{xcolor}
%\newcommand{\TC}[2] {\textcolor{red}{#2 \sout{#1}}}
%%\usepackage{soul}
%%\newcommand{\TCMode}[2] {\textcolor{red}{\st{#1} #2}}
\usepackage{cancel}
\newcommand{\TC}[2] {\textcolor{red}{\cancel{#1} #2}}


\usepackage{hyperref} % For hyperlinks in the document.
% Changes the form and colors of the hyperlinks of the document.
\hypersetup{
	colorlinks,% 
	citecolor=blue,% 
	filecolor=blue,% 
	linkcolor=blue,% 
	urlcolor=blue 
}

%%% Use Helvetica instead of Roman font
%\usepackage[T1]{fontenc}
%\usepackage{helvet}	
%\renewcommand*\familydefault{\sfdefault}

%\usepackage{subfig} 			% Sub-Figures
\usepackage{algorithmic} 		% For algorithms and pseudo-code 
%\usepackage{wrapfig}			% To wrap figures (in the license page
\usepackage{multirow} 

% A macro to insert two column wide figure
% Usage: \figuremacroTC{image_filename}{Caption header}{Caption}
% To reference the figure just use the image_filename:
% \ref{image_filename} will be a valid label
\newcommand{\figuremacroTC}[3]{
	\begin{figure*}[!t]%[htbp]
		\centering
		\includegraphics[width=1\textwidth]{#1}
		\caption[#2]{\textbf{#2} - #3}
		\label{#1}
	\end{figure*}
}

% A macro to insert Two Column wide figure
% But this time you can select the size of the figure
% Usage: \figuremacroTC{image_filename}{Caption header}{Caption}{RelativeSize}
% A relative Size of 1 means the figure has a width equal to textwidth
% To reference the figure just use the image_filename:
% \ref{image_filename} will be a valid label
\newcommand{\figuremacroTCW}[4]{
	\begin{figure*}[!t]%[htbp]
		\centering
		\includegraphics[width=#4\textwidth]{#1}
		\caption[#2]{\textbf{#2} - #3}
		\label{#1}
	\end{figure*}
}